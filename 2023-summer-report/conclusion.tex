\chapter{Conclusions}
We have modeled the formation of warm water vapor in protoplanetary disks with a comprehensive model. The radiative transfer of UV continuum and Ly$\alpha$ photons and the associated heating of dust grains are calculated with a Monte Carlo method. The density structure is described in a parameterized manner, and the gas temperature structure is solved based on the balance between heating and cooling mechanisms. The chemical evolution is followed for 1 Myr.

The final step is how to connect chemical compositions of gas and pebbles in disks with the compositions of young planets. This requires a better understanding of disk midplane compositions (see above), but also on how volatiles can be added to planets post-formation. The latter is especially important to predict the water and organic content of temperate, Earth-like planets. It is currently not clear under which conditions Earth-like planets can sample volatiles formed or preserved in outer disk regions, beyond the water snowline. More in-depth cometary studies are key to assess their formation zones, as well as their relationship to terrestrial volatiles. A frontier in the connection between astrochemistry and planet composition regards the carbon content of inner disks; in the ISM, 50\% of carbon resides in refractories, which appears preserved in comets, but seems to have been lost in the inner solar system. Depending on the nature of the refractory carbon removal mechanism, terrestrial planets may be generally carbon-poor, and depend on impacts both for water and organic delivery.

Finally, we note that Astrochemistry is an inherently interdisciplinary field. Its past and future successes depend on a combination of astronomical observations, chemical physics laboratory experiments, quantum calculations, molecular dynamics theory, and astrochemical models. We are entering an exciting era where astrochemistry is connecting with planetary and exoplanetary science to explore the formation of planets and the evolution of their hydrospheres and atmospheres. While the chemistry of planet formation sets the initial conditions of planets, the atmospheric chemistry and geochemistry determines how these initial conditions develop, and how often we may expect the complex chemistry we believe preceded life here on Earth.


