\chapter{Linking Chemistry to Planetary Formation}
To start with, snowline locations may affect the architectures of planetary systems. Snowlines of different volatiles are predicted to affect the grain coagulation rate, resulting in higher (or lower) rates of planet formation in the vicinity of snowline locations. In the Solar System, the presence of Jupiter at 5 au has long been associated with gas giant formation just outside the water snowline in the Solar Nebula (e.g., Stevenson \& Lunine 1988). There is also evidence for a pileup of gas giant exoplanets around 2 au (Fernandes et al. 2019), which may coincide with the location of the water snowline during the relevant disk evolutionary stage. Better exoplanet statistics, as well as observations of snowline locations in samples of disks, are needed to establish such links between planet formation and snowline locations with confidence.
\section{Chemical Inheritance}
Protostars form through the collapse of cloud cores. In our understanding of the chemistry of planet formation, this stage plays a key role because, first, it is during the warm-up of infalling cloud material towards the central protostar that much of the icy molecular cloud chemistry described in the previous section is revealed. Second, warm-up of interstellar grains in the protostellar envelope activates new chemical pathways that changes the compositions of the future solid building blocks of planetesimals and planets. Third, the protostellar disk that forms at this stage is the precursor to the planet-forming disks treated in the next section, and the balance between preservation or inheritance and reset at this stage provides the initial chemical conditions for planet formation. In this section we review the protostellar organic chemistry, and the chemistry or protostellar disks, after a brief review of the chemical structures of low-mass protostars (analogs to the protosun) and their surrounding environment.
\section{Composition of Planetary Atmospheres}
Most research linking disk and planet composition has focused on the formation of giant planets and the elemental composition of their atmospheres, but in the near future the study of Earth analog atmospheres will become feasible. The atmospheres and hydrospheres of rocky planets are shaped by a range of processes, including outgassing of magma, the length of a magma ocean phase, plate tectonics (e.g., Lichtenberg et al. 2022 and references therein), and impacts of meteorites and comets. The last two connect the disk molecular inventories with rocky planet compositions; therefore, comprehensive data sets on the distribution of key organics in asteroid and cometforming disk environments are needed to predict the prebiotic chemistry on young rocky planets.
\section{Isotopic Ratios}
Finally, isotopic ratios in gas and solids provide a tool to map the origins of cometary and planetary volatiles. So far, these ratios have been applied almost exclusively to the Solar System to constrain the origin of water on Earth and other planets, as well as in comets and asteroids (for reviews, see Ceccarelli et al. 2014, Altwegg et al. 2019). The first isotopic ratio in an exoplanet atmosphere was recently reported, potentially unlocking isotopic ratios as a complementary tool to elementary ratios when extracting a planet’s formation history (Y. Zhang et al. 2021). The deployment of this method, however, requires a detailed understanding of the isotopic composition and fractionation chemistry of disks, which is currently incomplete. Additional observations, modeling, and experiments are needed to establish a comprehensive interpretative framework for planetary volatile isotopic compositions.